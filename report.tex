\documentclass[10pt, a4paper,spanish]{article}

\usepackage[utf8]{inputenc}
\usepackage[spanish]{babel}

\usepackage[T1]{fontenc}

\usepackage[hmarginratio=1:1,top=32mm,columnsep=20pt]{geometry}
\usepackage[hang, small,labelfont=bf,up,textfont=it,up]{caption}


\usepackage{graphicx}
\graphicspath{ {images/} }

\usepackage{titlesec}
\renewcommand\thesection{\Roman{section}}
\renewcommand\thesubsection{\Roman{subsection}}
\titleformat{\section}[block]{\large\scshape\centering}{\thesection.}{1em}{}
\titleformat{\subsection}[block]{\large}{\thesubsection.}{1em}{}

\usepackage{fancyhdr}
\pagestyle{fancy}
\fancyhead{}
\fancyfoot{}
\fancyhead[C]{ \today \ $\bullet$ Minería de Datos $\bullet$ Lógica y Representación del Conocimiento}
\fancyfoot[RO]{\thepage}

%-------------------------------------------------------------------------------
%	TITLE SECTION
%-------------------------------------------------------------------------------

\title{\vspace{-15mm}\fontsize{24pt}{10pt}\selectfont\textbf{Lógica y Representación del Conocimiento}} % Article title

\author{Sergio García Prado}
\date{\today}

%-------------------------------------------------------------------------------

\begin{document}

	\maketitle % Insert title

	\thispagestyle{fancy} % All pages have headers and footers

%-------------------------------------------------------------------------------
%	TEXT
%-------------------------------------------------------------------------------

	\begin{center}
		\includegraphics[width=0.6\textwidth]{diagnostic-assistant}
	\end{center}

	\section{Elaborar una base de conocimiento para el asistente al diagnóstico en el dominio del cableado de una vivienda. Las reglas generales deben de permitir codificar la instancia específica que muestra la figura. Utilizar los principios generales para la elaboración de una ontología específica.}

		\paragraph{}
		La base de conocimiento que se ha elaborado permite comprobar el estado de todos los componentes del circuito, incluyendo cables, ya que se cree que estos también son suceptibles a fallos. Con esta ontología se podrá conocer los tipos de un elemento así como si tienen corriente o si lucen o producen electricidad(para enchufes).

		\paragraph{}
		Se ha decidido representar la base de conocimiento utilizando funciones por su mayor versatilidad a la hora de realizar preguntas una vez a pesar del mayor grado de dificultad a la hora de diseñar la ontología.

		\subsection{Vocabulario:}
			\begin{itemize}
				\item Constantes:
				\begin{itemize}
					\item $1$ y $0$: Utilizadas para  representan si hay corriente en un componente.
					\item $Up$ y $Down$: Utilizadas para representar si un conmutador está en una posición u otra.
					\item $CircuitBreaker$, $Switch$, $Light$, $PowerOutlet$, $Wire$: Utilizadas para diferenciar los distintos tipos de componentes.
				\end{itemize}
				\item Funciones:
				\begin{itemize}
					\item $in(x) = y \equiv$ La entrada de $x$ es $y$.
					\item $out(x,y) = z \equiv$ La salida $y$ de $x$ es  $z$.
					\item $signal(x) = y \equiv$ La señal de $x$ es  $y $.
					\item $position(x) = y \equiv$ El estado de $x$ es $y$.
					\item $type(x) = y \equiv$ El tipo de $x$ es $y$.
				\end{itemize}
				\item Predicados:
				\begin{itemize}
					\item $Lit(x) \equiv$ La variable $x$ luce.
					\item $Electrize(x) \equiv$ La variable $x$ produce electricidad (para enchufes).
					\item $Ok(x) \equiv$ La variable $x$ presenta un funcionamiento correcto.
					\item $Connected(x, y) \equiv$ La variable $x$ está conectada a la variable $y$.
				\end{itemize}
			\end{itemize}

		\subsection{Ontología general:}

			\begin{itemize}

				%types
				\item Restricciones de Tipos de Componentes:
				\begin{itemize}
					\item $ CircuitBreaker \neq Switch\neq Ligth \neq PowerOutlet \neq Wire   $
					\item $ \forall x [((type(x) = CircuitBreaker) \lor (type(x) = Switch) \lor (type(x) = Ligth) \lor (type(x) = PowerOutlet) \lor (type(x) = Wire))] $
				\end{itemize}
				% Connections
				\item Restricciones de Conexiones:
				\begin{itemize}
					\item $ 1 \neq 0 $
					\item $ \forall x [((signal(x) = 1) \lor (signal(x) = 0))] $
					\item $ \forall x \forall y [(Connected(x, y) \supset Connected(y, x))] $
					\item $ \forall x \forall y [(Connected(x, y) \land Ok(x) \land Ok(y) \supset (signal(x) = signal(y)))] $
				\end{itemize}

				% Switchess
				\item Restricciones de Conmutadores:
				\begin{itemize}
					\item $ Up \neq Down $
					\item $ \forall x [( (position(x) = Up) \lor (position(x) = Down))] $
					\item $ \forall x [( ( (type(x) = Switch) \land (position(x) = Up) ) \supset Connected(in(x), out(x, Up)))] $
					\item $ \forall x [( ( (type(x) = Switch) \land (position(x) = Down) ) \supset Connected(in(x), out(x, Down)))] $
					\item $ \forall x [( (type(x) = Switch)  \supset Connected(in(x), out(x)))] $
				\end{itemize}

				% Circuit Breakers
				\item Restricciones de diferenciales:
				\begin{itemize}
					\item $ \forall x \forall y \forall z [( ((type(y) = CircuitBreaker) \land Connected(x, y) \land Connected(y, z) ) \supset Connected(x, z))] $
				\end{itemize}

				% Lights
				\item Restricciones de Bombillas:
				\begin{itemize}
					\item$ \forall x [(((signal(x) = 1) \land (type(x) = Light) )  \supset Lit(x))] $
				\end{itemize}

				% Power Outlets
				\item Restricciones de Enchufes:
				\begin{itemize}
					\item$ \forall x [(((signal(x) = 1) \land (type(x) = PowerOutlet) )  \supset Electrize(x))] $
				\end{itemize}
			\end{itemize}

		\subsection{Ontología específica:}

			\begin{itemize}

				\item Constantes utilizadas para la ontología específica:
				\begin{itemize}
					\item $CB_i, i \in \{1,2\}$: Cada uno de los diferenciales de la figura.
					\item $S_i \in \{1,2,3\}$:  Cada uno de los conmutadores de la figura.
					\item $L_i \in \{1,2\}$: Cada una de las bombillas de la figura.
					\item $PO_i \in \{1,2\}$: Cada uno de los enchufes de la figura.
					\item $W_i \in \{0,1,2,3,4,5,6\}$: Cada uno de los cables de la figura.
					\item $OutsidePower$: Conexión de corriente externa de la figura.
				\end{itemize}
				\item Predicados que forman la ontología especifica:
				\begin{itemize}
					\item $ Ok(x),  x \in \{CB_i, S_j, L_k, PO_l, W_m, OutsidePower\}, i \in \{1,2\}, j \in \{1,2,3\},k \in \{1,2\},l \in \{1,2\},m \in \{1,2,3,4,5,6\}$
					\item $ type(CB_i) = CircuitBreaker, i \in \{1,2\} $
					\item $ type(S_i) = Switch, i \in \{1,2, 3\} $
					\item $ type(L_i) = Light, i \in \{1,2\} $
					\item $ type(PO_i) = PowerOutlet, i \in \{1,2\} $
					\item $ type(W_i) = Wire, i \in \{1,2, 3, 4, 5, 6\} $
					\item $ signal(OutsidePower) = 1 $
					\item $ position(S_1) = Down $
					\item $ position(S_2) = Up $
					\item $ position(S_3) = Up $
					\item $ Connected(OutsidePower, W_5)$
					\item $ Connected(W_5, CB_1)$
					\item $ Connected(W_5, CB_2)$
					\item $ Connected(CB_1, W_3)$
					\item $ Connected(CB_2, W_6)$
					\item $ Connected(W_6, PO_2)$
					\item $ Connected(W_3, PO_1)$
					\item $ Connected(W_3, in(S_1))$
					\item $ Connected(W_3, in(S_3))$
					\item $ Connected(out(S_3, Up), W_4)$
					\item $ Connected(W_4, L_2)$
					\item $ Connected(out(S_1, Up), W_1)$
					\item $ Connected(out(S_1, Down), W_2)$
					\item $ Connected(in(S_2), W_0)$
					\item $ Connected(W_0, L_1)$
				\end{itemize}
			\end{itemize}



	\clearpage
	\section{Partiendo de la base de conocimiento del asistente al diagnóstico que hemos utilizado en las prácticas de la asignatura, elaborar la ontología que la soporta. Compararla con la ontología elaborada en el problema anterior.}

		\paragraph{}
		Para el diseño de esta ontología se ha utilizado la base de conocimiento utilizada en las prácticas de prolog (se han añadido predicados auxiliares para que la ontolofía específica este formada tan solo por fórmulas atómicas). Esta es más limitada que la del ejercicio anterior, permitiendo comprobar el estado tan solo de los diferenciales, conmutadores y bombillas. La principal diferencia con respecto a la anterior (en cuanto a estructura y no a alcance del problema) es que esta ontología  tan solo utiliza predicados para representar el conocimiento, lo cual limita en cierto grado la capacidad expresiva y dificulta la formulación de prefuntas.

		\subsection{Vocabulario:}
			\begin{itemize}
				\item Predicados:
				\begin{itemize}
					\item $Light(x) \equiv$ La variable $x$ es una bombilla.
					\item $Lit(x) \equiv$ La variable $x$ luce.
					\item $Ok(x) \equiv$ La variable $x$ presenta un funcionamiento correcto.
					\item $Connected(x, y) \equiv$ La variable $x$ está conectada a la variable $y$.
					\item $ConnectedOk(x, y, z) \equiv$ La variable $x$ está conectada a la variable $y$ si la variable $z$ presenta un funcionamiento correcto.
					\item $ConnectedOkUp(x, y, z) \equiv$ La variable $x$ está conectada a la variable $y$ si la variable $z$ presenta un funcionamiento correcto y está en el estado $Up$.
					\item $ConnectedOkDown(x, y, z) \equiv$ La variable $x$ está conectada a la variable $y$ si la variable $z$ presenta un funcionamiento correcto y está en el estado $Down$.
					\item $Connected(x, y) \equiv$ La variable $x$ está conectada a la variable $y$.
					\item $Live(x) \equiv$ La variable $x$ tiene corriente.
					\item $Up(x) \equiv$ La variable $x$ en el estado up.
					\item $Down(x, y) \equiv$ La variable $x$ está en el estado down.
				\end{itemize}
			\end{itemize}

		\subsection{Ontología general:}

			\begin{itemize}
				\item $ \forall x [(Light(x) \land Ok(x) \land Live(x) \supset Lit(x)] $
				\item $ \forall x \forall y [(Connected(x, y) \land Live(x) \supset Live(y)] $
				\item $ \forall x \forall y \forall z [(ConnectedOk(x, y) \land Ok(z) \supset Connected(x, y)] $
				\item $ \forall x \forall y \forall z [(ConnectedOkUp(x, y) \land Ok(z) \land Up(z) \supset Connected(x, y)] $
				\item $ \forall x \forall y \forall z [(ConnectedOkDown(x, y) \land Ok(z) \land Down(z) \supset Connected(x, y)] $

			\end{itemize}

		\subsection{Ontología específica:}

			\begin{itemize}
				\item Constantes utilizadas para la ontología específica:
				\begin{itemize}
					\item $CB_i, i \in \{1,2\}$: Cada uno de los diferenciales de la figura.
					\item $S_i, i \in \{1,2,3\}$:  Cada uno de los conmutadores de la figura.
					\item $L_i, i \in \{1,2\}$: Cada una de las bombillas de la figura.
					\item $P_i, i \in \{1,2\}$: Cada uno de los enchufes de la figura.
					\item $W_i, i \in \{0,1,2,3,4,5,6\}$: Cada uno de los cables de la figura.
					\item $Outside$: Conexión de corriente externa de la figura.
				\end{itemize}
				\item Predicados que forman la ontología especifica:
				\begin{itemize}
				 	\item $ Ok(x),  x \in \{CB_i, S_j, L_k\}, i \in \{1,2\}, j \in \{1,2,3\},k \in \{1,2\}$
					\item $ Live(Outside)$
					\item $ Light(L_1)$
					\item $ Light(L_2)$
					\item $ Down(S_1)$
					\item $ Up(S_2)$
					\item $ Up(S_3)$
					\item $ Connected(L_1, W_0)$
					\item $ Connected(L_2, W_4)$
					\item $ Connected(P_1, W_3)$
					\item $ Connected(P_2, W_6)$
					\item $ Connected(W_5, Outside)$


					\item $ ConnectedOk(W_2, W_5, CB_1)$
					\item $ ConnectedOk(W_6, W_5, CB_2)$

					\item $ ConnectedOkUp(W_1, W_3, S_1$
					\item $ ConnectedOkDown(W_2, W_3, S_1)$

					\item $ ConnectedOkUp(W_0, W_1, S_2$
					\item $ ConnectedOkDown(W_0, W_2, S_2)$


					\item $ ConnectedOkUp(W_4, W_3, S_3)$
				\end{itemize}

			\end{itemize}

\end{document}
